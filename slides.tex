%%%%%%%%%%%%%%%%%%%%%%%%%%%%%%%%%%%%%%%%%%%%%%%%%%%%%%%%%%%%%%%
%
% Welcome to Overleaf --- just edit your LaTeX on the left,
% and we'll compile it for you on the right. If you open the
% 'Share' menu, you can invite other users to edit at the same
% time. See www.overleaf.com/learn for more info. Enjoy!
%
%%%%%%%%%%%%%%%%%%%%%%%%%%%%%%%%%%%%%%%%%%%%%%%%%%%%%%%%%%%%%%%


% Inbuilt themes in beamer
\documentclass{beamer}
\usepackage[utf8]{inputenc}
\usepackage{amsmath}
\usepackage{amsfonts}
\usepackage{mathtools}
\usepackage{amssymb}
\providecommand{\pr}[1]{\ensuremath{\Pr\left(#1\right)}}
\providecommand{\brak}[1]{\ensuremath{\left(#1\right)}}
\newcommand{\Int}{\int\limits}



% Theme choice:
\usetheme{CambridgeUS}

% Title page details: 
\title{AI1110 Assignment-12} 
\author{Gollapudi Sasank CS21BTECH11019}
\date{\today}
\logo{\large \LaTeX{}}


\begin{document}

% Title page frame
\begin{frame}
    \titlepage 
\end{frame}

% Remove logo from the next slides
\logo{}


% Outline frame
\begin{frame}{Outline}
    \tableofcontents
\end{frame}


% Lists frame
\section{Question}
\begin{frame}{Question}
In a production process the number of defective units per hour is a Poisson distributed random variable $x$ with parameter $\lambda = 5$ . A new process is is introduced and it is observed that the hourly defectives in a 22-hour period are \\
$x_i = 3,0,5,4,2,6,4,1,5,3,7,4,0,8,3,2,4,3,6,5,6,9$ \\
Test the hypothesis $ \lambda = 5 $ against $ \lambda < 5$ with $ \alpha = 0.05 $
\end{frame}

\section{Solution}
\begin{frame}{Solution}
We shall use the sum of $x_i$ as Test Static ($q$) \\
$ q = x_1 + x_2 + ...... + x_n$ \\
Here $q$ is also a poisson random variable with parameter $\eta_q = n\lambda$ \\
we need to test the hypothesis  $H_0$ ($ \lambda = 5 $) \\
Under Hypothesis $H_0$ , $\lambda = \lambda_0 = 5$ \\
The critical region of the hypothesis is $q<q_\alpha$ , where \\
$ q = x_1 + x_2 + ...... + x_n = 90$ \\
To find $q_\alpha$ we use the normal approximation method with $\alpha = 0.05$ \\
\begin{align}
q_\alpha &= n\lambda_0 + z_\alpha \sqrt{n\lambda_0} \\
\end{align}
\end{frame}

\begin{frame}
Here $ n=22,\lambda_0 = 5,\alpha = 0.05 $ ,
\begin{align}
\Rightarrow z_\alpha &= z_{0.05} \\
\Rightarrow z_\alpha &= -z_{1-0.05} \\
\Rightarrow z_\alpha &= -z_{0.95} \\
\Rightarrow z_\alpha &= -1.645
\end{align}
\begin{align}
q_\alpha &= 110 - (1.645)(\sqrt{110}) \\
\Rightarrow q_\alpha &= 110 - 17.25 \\
\Rightarrow q_\alpha &= 92.75
\end{align}
Here the Hypothesis $H_1$ is $\lambda < \lambda_0$ \\
We accept $H_0$ iff $q > q_\alpha$. \\
Here $ q < q_\alpha $. \\
So we reject the hypothesis $H_0$.
\end{frame}
\end{document}